\documentclass[pdftex,12pt,a4paper]{article}

\usepackage{wrapfig, amssymb, amsmath, graphicx, subfigure, pifont, seqsplit}
\usepackage[dutch]{babel}
\usepackage[top=0.5in, bottom=0.5in, left=1in, right=1in]{geometry}
\pagenumbering{arabic}
\newcommand{\HRule}{\rule{\linewidth}{0.5mm}}

\begin{document}
\begin{titlepage}

% Upper part of the page
\begin{flushleft}
\includegraphics[trim=0mm 0mm 0mm 0mm, width=1\textwidth]{./logo.jpg}\\
\end{flushleft}
\begin{center}
	\textsc{\Large Informatie en Communicatie}\\[0.5cm]

    % Title
    \HRule \\[0.4cm] { \huge \bfseries HW-1}\\[0.4cm]

    \HRule \\[1.5cm]

    % Author and supervisor
\begin{minipage}{0.4\textwidth}
\begin{flushleft} \large \emph{Authors:}\\
Abe \textsc{Wiersma}\\
\end{flushleft}
\end{minipage}
\begin{minipage}{0.4\textwidth} \begin{flushright} \large \end{flushright}\end{minipage}

    \vfill

    % Bottom of the page 
    {\large \today}

\end{center}
\end{titlepage}
\pagebreak

\begin{enumerate}
    \item
        \ding{51}
    \item
        \begin{enumerate}
            \item
                Have 52 choices..draw a card..have 51 choices..draw a card....have 1 choice.\\

                $ 52! = 
                    \seqsplit{%
                    80658175170943878571660636856403766975289505440883277824000000000000}\\ = 8.0658175170943878571660636856403766975289505440883277824... × 10^{67}$
            \item
                Have 52 choices out of 104 cards..draw a card..have 52 choices out of a 103 cards..draw a card..have either 51 choices out of a 102 cards or 52 choices out of a 102 cards..draw a card....have 1 choice.\\

                $104! / 2^{52}\\ = \seqsplit{%
                2286841104292142739817480260270397947731385025482502766874913064240686762501672463119886085466145431873911265472729321167650816000000000000000000000000}\\ = 2.2868411042921427398174802602703979477313850254825027668... × 10^{150}$
        \end{enumerate}
    \item
        Prove  the  following  inequality  for  real  numbers $p_1,p_2,\dots,p_n \in [0,1]$\\
        (i.e. $0 \leq p_i \leq 1$ $i = 1,2\dots,n$):\\
        (*) $$(1-p_1)(1-p_2)\dots(1-p_n) \geq 1 - p_1 - p_2 - \dots - p_n$$
        \textbf{Left side}
            \begin{align}
                (1-p_1)(1-p_2)\dots(1-p_n) &=(1-\mathbb{P}(E_1))(1-\mathbb{P}(E_2))\dots(1-\mathbb{P}(E_n))\\
                &=(\mathbb{P}(\overline{E_1}))(\mathbb{P}(\overline{E_2}))\dots(\mathbb{P}(\overline{E_n}))\\
                &=\prod_i^n\mathbb{P}(\overline{E_i})\\
                \intertext{Multiplication of independent events is same as the intersection of these independent events.}
                &=\mathbb{P}\left(\bigcap_i^n(\overline{E_i})\right)
            \end{align}
        \textbf{Right side}
            \begin{align}
                1 - p_1 - p_2 - \dots - p_n &= 1-\sum_i^np_i\\
                                            &=1-\sum_i^n\mathbb{P}(E_i)\\
                                            &=1-\sum_i^n(1 - \mathbb{P}(\overline{E_i}))\\
                                            &=1-n-\sum_i^n(\mathbb{P}(\overline{E_i}))
            \end{align}
        \newpage
        \textbf{Combined}
            \begin{align}
                \mathbb{P}\left(\bigcap_i^n(\overline{E_i})\right) &\geq 1-n-\sum_i^n(\mathbb{P}(\overline{E_i}))\\
                n &\geq 1 - \sum_i^n(\mathbb{P}(\overline{E_i})) - \mathbb{P}\left(\bigcap_i^n(\overline{E_i})\right)\\
                n &\geq 1 - \mathbb{P}\left(\bigcup_i^n(\overline{E_i})\right)
            \end{align}
        (*new)
            $$\mathbb{P}\left(\bigcup_i^n(\overline{E_i})\right) + n\geq 1$$
        (n=1)
            $$\mathbb{P}(\overline{E_1}) + 1 \geq 0$$
            $$\mathbb{P}(\forall \chi) \in [0,1]$$
            $$\textit{so \ding{51}}$$
        (n)
            $$\mathbb{P}\left(\bigcup_i^n(\overline{E_i})\right) + n\geq 1 $$
            $$\textit{Since}$$
            $$\mathbb{P}(\forall \chi) \in [0,1]$$
            $$n \in {N}$$
            $$ \textit{so \ding{51}}$$
        (*new) holds and so does (*)
    \item
        \begin{enumerate}
            \item
                \begin{align}
                    \operatorname{Var}(X) &= \mathbb{E}[(X - \mu)^2] \\
                                          &= \mathbb{E}[(X - \mathbb{E}[X])^2] \\
                                          &= \mathbb{E}[X^2 - 2X\mathbb{E}[X] + (\mathbb{E}[X])^2] \\
                                          &= \mathbb{E}[X^2] - 2\mathbb{E}[X]\mathbb{E}[X] + (\mathbb{E}[X])^2 \\
                                          &= \mathbb{E}[X^2 ] - (\mathbb{E}[X])^2
                \end{align}
            \item
                \begin{align}
                    \operatorname{Var}(aX) &= \operatorname{Var}(Y)\\
                                           &= \mathbb{E}[Y^2] - (\mathbb{E}[Y])^2\\
                                           &= \mathbb{E}[aX^2] - (\mathbb{E}[aX])^2\\
                                           &= a^2\mathbb{E}[X^2] - (a\mathbb{E}[X])^2\\
                                           &= a^2\mathbb{E}[X^2] - a^2(\mathbb{E}[X])^2\\
                                           &= a^2(\mathbb{E}[X^2] - (\mathbb{E}[X])^2)\\
                                           &= a^2\operatorname{Var}(aX)
                \end{align}
                \begin{align}
                    \operatorname{Var}(X + a) &= \operatorname{Var}(Y)\\
                                              &= \mathbb{E}[Y^2] - (\mathbb{E}[Y])^2\\
                                              &= \mathbb{E}[(X+a)^2] - (\mathbb{E}[(X+a)])^2\\
                                              &= \mathbb{E}[X^2 + a^2 + 2aX] - (\mathbb{E}[X] + a)^2\\
                                              &= \mathbb{E}[X^2]+ a^2 + 2a\mathbb{E}=[X] - (\mathbb{E}[X]^2 + a^2 + 2a\mathbb{E}[X])\\
                                              &= \mathbb{E}[X^2]+ a^2 + 2a\mathbb{E}=[X] - (\mathbb{E}[X])^2 - a^2 - 2a\mathbb{E}[X]\\
                                              &= \mathbb{E}[X^2] - (\mathbb{E}[X])^2
                \end{align}
            \item
                \begin{align}
                    \operatorname{Var}(X + Y) &= \operatorname{Var}(Z)\\
                                              &= \mathbb{E}[Z^2] - (\mathbb{E}[Z])^2\\
                                              &= \mathbb{E}[(X+Y)^2] - (\mathbb{E}[(X+Y)])^2\\
                                              &= \mathbb{E}[X^2+Y^2+2XY] - (\mathbb{E}[X]+\mathbb{E}[Y])^2\\
                                              &= \mathbb{E}[X^2]+\mathbb{E}[Y^2]+2\mathbb{E}[XY] - (\mathbb{E}[X]^2+\mathbb{E}[Y]^2+2\mathbb{E}[X]\mathbb{E}[Y])\\
                                              &= \operatorname{Var}(X) + \operatorname{Var}(Y) + 2(\mathbb{E}(XY)-\mathbb{E}(X)\mathbb{E}(Y))\\
                                              &= \operatorname{Var}(X) + \operatorname{Var}(Y) + 2\operatorname{Cov}(X,Y)\\
                                              \intertext{Covariance of two independent variables is 0}
                                              &= \operatorname{Var}(X) + \operatorname{Var}(Y)
                \end{align}
            \item Let $X$ be a random variable with \textit{Bernoulli distribution}
                \begin{align}
                    \mathbb{E}[X] &= \sum_{x\in X}x\mathbb{P}_x(x)\\
                                  &= 1*\mathbb{P}_x(1) + 0*\mathbb{P}_x(0)\\
                                  &= 1*p + 0*(1-p)\\
                                  &= p
                \end{align}
                \begin{align}
                    \operatorname{Var}[X] &= \mathbb{E}[X^2 ] - (\mathbb{E}[X])^2\\
                                          &= \left(\sum_{x\in X}x^2P_x(x)\right) - p^2\\
                                          &= 1^2*\mathbb{P}_x(1) + 0^2*\mathbb{P}_x(0) - p^2\\
                                          &= 1*p + 0*(1-p) - p^2\\
                                          &= p - p^2\\
                                          &= p(1-p)
                \end{align}
            \newpage
            \item Let $Y$ be a random variable with \textit{binomial distribution} $P_Y(y) = \tbinom ny p^y(1-p)^{n-y}$
                \begin{align}
                    \mathbb{E}[Y] &= \mathbb{E}\left[\sum_n^i X_i\right]\\
                                  \intertext{Expected value is the expected value of the sum of the independent \textit{Bernoulli distributions}.}
                                  &= \sum_n^i\mathbb{E}[X_i]\\
                                  &= \sum_n^i p\\
                                  &= np
                \end{align}
                \begin{align}
                    \operatorname{Var}[Y] &= \operatorname{Var}\left[\sum_n^i X_i\right]\\
                                          &= \sum_n^i\operatorname{Var}[X_i]\\
                                          &= \sum_n^i p(1-p)\\
                                          &= n p(1-p)
                \end{align}
        \end{enumerate}
    \item
        An urn contains $K$ balls, of which $B$ are black and $W = K - B$ balls are replaced when drawn.\\
        Classic \textit{Bernoulli distribution}
        \begin{enumerate}
            \item 
                $f(n_b; N, B/K) = f(k; n, p) = \tbinom nk p^k(1-p)^{n-k}$
            \item
                A Bernoulli distribution has $\mathbb{E}[X] = np$, this one has $n = N$ and $p = B/K$\\
                so the expected value $= N * (B/K)$\\
                N=5 and N=400 for B=2 and K=10
                \begin{enumerate}
                    \item
                        $5*(2/10) = 1$\\
                    \item
                        $400*(2/10) = 80$\\
                \end{enumerate}

                A Bernoulli distribution has $\operatorname{Var}[X] = np(1-p)$\\
                so the variance $= N * (B/K)(1 - (B/K))$\\
                N=5 and N=400 for B=2 and K=10
                \begin{enumerate}
                    \item
                        $5*0.2(1-0.2) = 0.8$\\
                    \item
                        $400*0.2(1-0.2) = 64$\\
                \end{enumerate}
        \end{enumerate}

\end{enumerate}

\end{document}
