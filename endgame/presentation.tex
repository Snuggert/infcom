%-------------------------------------------------------------------------------
%   PACKAGES EN CONFIGURATIE
%-------------------------------------------------------------------------------

\documentclass{uva-inf-presentation}
\usepackage[dutch]{babel}
\uselanguage{Dutch}
\languagepath{Dutch}

%-------------------------------------------------------------------------------
%   TITELPAGINA EN INHOUDSOPGAVE
%-------------------------------------------------------------------------------

% Een korte titel voor de footer, een langere titel voor de titelpagina
\title[Breaking Traditional Ciphers]{\small{Information and Communication}\\\textbf{Codebreaking for Traditional Cipher Systems}}

% Vul beide namen hier in
\author{Abe Wiersma}
\institute[UvA]{\includegraphics[width=6cm]{logoUvA_nl}}
\date{\today}

\begin{document}

\begin{frame}
% De eerste slide is de titelpagina
\titlepage
\end{frame}

\begin{frame}
% Geeft een inhoudsopgave voor de presentatie
\frametitle{Inhoudsopgave}
\tableofcontents
\end{frame}

%-------------------------------------------------------------------------------
%   PRESENTATIE SLIDES
%-------------------------------------------------------------------------------

% Secties zijn er om structuur in je presentatie aan te brengen
% en worden direct in de inhoudsopgave opgenomen
\section{Introduction in Ciphers}
%################################################
% Maak een subsection voor een serie slides met een gedeeld thema
\subsection{Modern Ciphers}
%------------------------------------------------
\begin{frame}
\frametitle{Modern Ciphers}

\begin{itemize}
    \item Private-key cryptography, where the same key is used for encryption
        and decryption.
    \item Public-key cryptography, where two different keys are used for
        encryption and decryption.
\end{itemize}
Ciphers can be distinguished into two types by the type of input data:
\begin{itemize}
    \item Block ciphers, which encrypt block of data of fixed size
    \item Stream ciphers, which encrypt continuous streams of data
\end{itemize}
\end{frame}

%------------------------------------------------
\subsection{Traditional Ciphers}
%------------------------------------------------

\begin{frame}
\frametitle{Traditional Ciphers}
\begin{itemize}
    \item Origin lies in ancient Egypt $\sim4000$ years ago. %mostly army/government
    \item Ends after second world war with the emergence of the computers.
\end{itemize}
Uses
\begin{itemize}
    \item War movement.
    \item Government Secrets.
\end{itemize}
\end{frame}

\section{Breaking Traditional Ciphers}

\begin{frame}
\frametitle{Types: Traditional Ciphers}
\begin{itemize}
    \item Substitution Cipher
    \item Permutation Cipher
    \item Running key Cipher
    \item An Enigma-style periodic polyalphabetic Cipher
\end{itemize}
\end{frame}

\subsection{Substitution Cipher}

\begin{frame}
\frametitle{Substitution Cipher:\\ Example}
Example cipher
\end{frame}

\begin{frame}
\frametitle{Substitution Cipher:\\ Problem}
Example cipher
\end{frame}

\begin{frame}
\frametitle{Substitution Cipher:\\ Solution}
Example cipher
\end{frame}


%------------------------------------------------
%################################################
\section{}
%################################################
%------------------------------------------------
\begin{frame}
\frametitle{Stelling}
\begin{theorem}[Massa-energierelatie]
$E = mc^2$
\end{theorem}
\end{frame}
%------------------------------------------------
\begin{frame}
\frametitle{Conclusie}
\Large{\centerline{Belangrijkste conclusie}}
\end{frame}
%------------------------------------------------
%################################################
\end{document}